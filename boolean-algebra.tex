\documentclass{article}

\usepackage{tikz}
\usepackage[margin=1in, includefoot]{geometry}
\usepackage[fleqn]{amsmath}
\usepackage{xcolor}
\usepackage{fancyhdr}
\usepackage{listings}

\setlength\parindent{0pt}
\pagestyle{fancy}
\linespread{2}
\title{Boolean Algebra}
\date{}

\begin{document}
\maketitle

George Boole, 19th century mathematician, developed a mathematical system involving logic..
\section*{Axioms \& Theorem}

\subsection*{Laws}
\textbf{complementarity} \\
\textbf{laws of zeros and ones} \\
\textbf{identities} \\
\textbf{idempotent law} \\
\textbf{communitive law} \\
\textbf{associative} \\
\textbf{distribution}
  \begin{flalign*}
    & x + y \cdot z = (x + y) \cdot (x + z) \\
    & x \cdot (y + z) = x \cdot y + x \cdot z
  \end{flalign*}
\textbf{uniting theorem} \\
\textbf{de morgan's law}
\subsection*{Common Math Equation}
\begin{flalign*}
& x + \bar{x} \cdot y = x + y \\
& x \cdot y + y \cdot z + \bar{x} \cdot z = x \cdot y + \bar{x} \cdot z \\
& x \cdot z + x \cdot \bar{z} = x \\
& (x + z) \cdot (x + \bar{z}) = x
\end{flalign*}

\subsection*{Duality}
Given a logic expression, its dual is obtained by replacing all + operators with · operators, and vice versa, and by replacing all 0s with 1s, and vice versa. The dual of any true statement (axiom or theorem) in Boolean algebra is also a true statement.
\begin{flalign*}
  & x \cdot z + x \cdot \bar{z} = x \\
  & (x + z) \cdot (x + \bar{z}) = x
\end{flalign*}

\subsection*{XOR}
\subsubsection*{Properties}
  \begin{flalign*}
    & x \land (y \oplus z) = (x \land y) \oplus (x \land z)
  \end{flalign*}

\subsection*{Other Basic Gates}
NAND: $\bar{A} + \bar{B}$ \\
NOR: $\bar{A}\bar{B}$ \\
XOR: $\bar{A}B + A\bar{B}$ \\
XNOR: $AB + \bar{A}\bar{B}$

\section*{Logic Networks}

\subsection*{NAND \& NOR Logic Networks}
Consider a network which consists of only AND and OR logic gates. 
This network can be transformed into a network of only NAND gates. 
First, each connection between the AND gate and an OR gate is replaced by a connection that includes two inversions of the signal: 
one inversion at the output of the AND gate and the other at the input of the OR gate.
Such double inversion has no effect on the behavior of the network, the OR gate with inversions at its inputs is equivalent to a NAND gate. 
Thus we can redraw the network using only NAND gates.
\begin{flalign*}
f & = x_1 \cdot x_2 + x_3 \cdot x_4 \\
  & = \overline{\overline{x_1 \cdot x_2}} + \overline{\overline{x_3 \cdot x_4}} \\
  & = \textnormal{NAND}(\overline{x_1 \cdot x_2}, \; \overline{x_3 \cdot x_4}) \\
  & = \textnormal{NAND}(\textnormal{NAND}(x_1, \, x_2), \; \textnormal{NAND}(x_3, \, x_4))
\end{flalign*}

Using NAND gates can implement SOP (sum-of-products) networks. 
Similarly, using NOR gates can implement a POS (product-of-sums) networks.

\section*{Minimization \& Karnaugh Maps}

\end{document}